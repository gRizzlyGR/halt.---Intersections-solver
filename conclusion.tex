\chapter{Conclusioni}

Il progetto ha proposto un risolutore di incroci, ritrovabili sugli esami della patente di guida e su eserciziari correlati. È stato definito un formalismo logico che spiegassi tutti gli oggetti coinvolti (veicoli, bracci, segnali ecc.), trasferendo poi la conoscenza in un software Prolog, arricchito anche da una componente grafica con l'ausilio del Java.

I risultati sono soddisfacenti: solo 2 incroci su 58 non risultano conformi alle regole, quindi in futuro sarà necessario espandere e modificarle per coprire anche questi casi.

Un domani si potrebbe anche realizzare un modulo di riconoscimento e caricamento di un incrocio partendo direttamente dalle immagini, senza dover trascrivere i fatti a mano, che può risultare tedioso.