\chapter{Analisi}

Dopo aver studiato il dominio su varie fonti, è possibile estrapolare dei requisiti utili per soddisfare l'obiettivo del progetto.

\section{Raccolta dei requisiti}
La raccolta è avvenuta consultando diversi codici, articoli di legge e libri che si interessano al dominio. Le parti significative sono disponibili nel capitolo precedente di studio del dominio.

\section{Analisi dei requisiti}
Un'\emph{intersezione} è formata da tre o più \emph{tronchi} stradali convergenti nello stesso punto. Su ogni tronco è possibile trovare un \emph{segnale} che lo regola. I \emph{veicoli} devono rispettare le norme di \emph{precedenza} imposte dal segnale o, in mancanza, da ciò che è scritto nel codice, per esempio un veicolo deve sempre dare la precedenza a chi viene da \emph{destra}, salvo diversa segnalazione. Alcune tipologie di veicoli godono di deroghe ai vincoli di precedenza, ad esempio \emph{veicoli in allarme} come \emph{volanti delle forze dell'ordine}, \emph{camion dei vigili del fuoco} o \emph{ambulanze}, oppure \emph{veicoli su rotaia} come i \emph{tram}.

\section{Glossario}
Qui vengono riportati i concetti chiave del dominio e gli eventuali sinonimi:
\begin{itemize}
	\item Intersezione = Incrocio.
	\item Tronco = Braccio.
	\item Precedenza.
	\item Destra.
	\item Veicolo in allarme = Veicolo in soccorso.
	\item Volante delle forze dell'ordine.
	\item Camion dei vigili del fuoco = Autopompa.
	\item Ambulanza.
	\item Veicolo su rotaia = Tram.
\end{itemize}
