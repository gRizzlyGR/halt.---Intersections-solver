\chapter{Studio del dominio}

\section{Intersezione}
Si definisce intersezione stradale (nodo) l'area individuata da tre o più tronchi stradali (archi) che convergono in uno stesso punto, nonché dai dispositivi e dagli apprestamenti atti a consentire ed agevolare le manovre per il passaggio da un tronco all'altro.
Le intersezioni, qualunque sia la loro localizzazione territoriale, costituiscono punti critici del sistema viario per effetto delle interferenze che in esse si instaurano tra le diverse correnti di traffico.

\section{Segnali}
Il segnale stradale serve ad indicare una prescrizione, un avvertimento o un'indicazione a tutti i veicoli circolanti e ad ogni altro utente della strada.

Tramite la segnaletica il gestore di una strada (persona fisica o giuridica) comunica agli utenti la disciplina della circolazione: regole, pericoli, indicazioni ed informazioni utili. Per conseguire l'abilitazione alla guida di veicoli (patente di guida) è richiesto obbligatoriamente imparare quali sono i segnali, come riconoscerli e soprattutto cosa significano. Senza l'apposizione della segnaletica non possono essere correttamente applicate le regole della circolazione stradale. Il linguaggio della segnaletica stradale è uno dei più diffusi al mondo, ciò fa sì gli utenti di tutto il mondo sappiano il significato di una figura ottagonale con la scritta STOP, così come sappiano cosa voglia dire la luce rossa di un semaforo.

Il complesso della segnaletica stradale viene suddiviso in cinque tipologie generali\cite{CdsTitIICapoII}, come descritto di seguito:

\begin{itemize}

\item \textbf{Segnaletica manuale}, ossia le segnalazioni date dagli organi di polizia stradale (polizia locale, Polizia di Stato, Carabinieri ecc.)

\item \textbf{Segnali luminosi}, caratterizzati dalla possibilità di fornire maggiore impatto visivo e/o informazioni dinamiche, vengono suddivisi in:
\begin{itemize}
	\item segnali di pericolo e di prescrizione;
	\item segnali di indicazione;
	\item tabelloni luminosi rilevatori della velocità in tempo reale dei veicoli in transito;
	\item lanterne semaforiche veicolari normali;
	\item lanterne semaforiche veicolari di corsia;
	\item lanterne semaforiche veicolari per corsie reversibili;
	\item lanterne semaforiche per i veicoli di trasporto pubblico;
	\item lanterne semaforiche pedonali;
	\item lanterne semaforiche per velocipedi;
	\item lanterna semaforica gialla lampeggiante;
	\item lanterne semaforiche speciali;
	\item segnali luminosi particolari.

\end{itemize}
\item \textbf{Segnali verticali}, a loro volta sono suddivisi in:
\begin{itemize}
	\item segnali di pericolo - preavvisano l'esistenza di pericoli;
	\item segnali di prescrizione - notificano obblighi, divieti e limitazioni e vengono indicati come:.
\begin{itemize}
	\item segnali di precedenza;
\item segnali di divieto;
\item segnali di obbligo;
\end{itemize}
\item segnali di indicazione che forniscono informazioni utili o necessarie per la guida, suddivisi a loro volta in:
\begin{itemize}
	\item segnali di preavviso;
	\item segnali di direzione;
	\item segnali di conferma;
	\item segnali di identificazione strade e progressiva distanziometrica;
	\item segnali di itinerario;
	\item segnali di località e centro abitato;
	\item segnali di nome strada;
	\item segnali turistici e di territorio;
	\item altri segnali che danno informazioni necessarie per la guida dei veicoli;
	\item altri segnali che indicano installazioni o servizi.

\end{itemize}\end{itemize}

\item \textbf{Segnali orizzontali}, sono quelli tracciati sulla strada, e si suddividono in:
\begin{itemize}
	\item linea trasversale d'arresto;
	\item strisce longitudinali;
	\item strisce trasversali;
	\item attraversamenti pedonali o ciclabili;
	\item frecce direzionali;
	\item iscrizioni e simboli;
	\item strisce di delimitazione degli stalli di sosta o per la sosta riservata;
	\item isole di traffico o di presegnalamento di ostacoli entro la carreggiata;
	\item strisce di delimitazione della fermata di veicoli in servizio di trasporto pubblico di linea;
	\item altri segnali stabiliti dal regolamento.
\end{itemize}

\item \textbf{Segnali e attrezzature complementari}, destinati a evidenziare particolari situazioni, vengono utilizzati sul tracciato stradale, nelle immediate vicinanze di particolari curve o punti critici, per segnalare ostacoli sposti sulla carreggiata e per impedire la sosta o rallentare la velocità (es. dossi artificiali).
\end{itemize}

\section{Veicoli}

\section{Situazioni di precedenza}
