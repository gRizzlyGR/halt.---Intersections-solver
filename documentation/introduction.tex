\chapter{Introduzione}

\vspace{-4em}

\begin{figure}[!htb]
	\centering
	\includegraphics[scale=.15]{images/halt}
\end{figure}

Il presente documento mira a descrivere \fbox{\includegraphics[scale=.15]{images/name}}, il lavoro di Intelligenza Artificiale relativo alla costruzione di un sistema scritto in \emph{Prolog} per la risoluzione di incroci, tipici nei quiz degli esami per l'ottenimento della patente di guida. Il programma dovrà quindi essere in grado di restituire l'ordine di circolazione dei veicoli coinvolti dopo aver ottenuto in ingresso una descrizione dell'incrocio.

Per come si presentano veicoli, segnali e bracci all'interno di un incrocio, è intuitivo descrivere queste componenti tramite relazioni di oggetti, utilizzando un formalismo logico che, in questo caso, risulta molto più comodo rispetto ad un linguaggio imperativo. 

Il lavoro si svolgerà in diverse fasi, dallo studio del dominio alla creazione del software, fino al collaudo su esempi reali.
