\chapter{Analisi}

Dopo aver studiato il dominio su varie fonti, è possibile estrapolare dei requisiti utili per soddisfare l'obiettivo del progetto.

\section{Raccolta dei requisiti}
La raccolta è avvenuta consultando diversi codici, articoli di legge, libri e manuali che si interessano al dominio. Le parti significative sono disponibili nel capitolo precedente di studio del dominio.

\section{Vincoli}
Gli incroci presi in considerazione sono quelli in cui si richiede all'esaminando di risolverli secondo le regole del CdS, soprattutto per le persone che vogliono ottenere patenti di tipo B. Altri tipi di incrocio o simili vengono tralasciati, ad esempio i casi di attraversamento tranviario dove i tram sono regolati da una propria segnaletica, oppure incroci in cui si chiede di valutare la correttezza della manovra dei veicoli interessati.

\section{Analisi dei requisiti}
Un'intersezione è formata da tre o più bracci stradali convergenti nello stesso punto. Su ogni tronco è possibile trovare un segnale che lo regola. I veicoli devono rispettare le norme di precedenza imposte dal segnale o, in mancanza, da ciò che è scritto nel codice, per esempio un veicolo deve sempre dare la precedenza a chi viene da destra, salvo diversa segnalazione. Alcune tipologie di veicoli godono di deroghe ai vincoli di precedenza, ad esempio veicoli in allarme come volanti delle forze dell'ordine, camion dei vigili del fuoco o ambulanze, oppure veicoli su rotaia come i tram.

\section{Glossario}
\label{sec:glossary}
Qui vengono riportati i concetti chiave del dominio e gli eventuali sinonimi:
\begin{itemize}
	\item Intersezione = Incrocio.
	\item Tronco = Braccio.
	\item Veicolo.
	\item Precedenza.
	\item Destra.
	\item Veicolo in allarme = Veicolo in soccorso.
	\item Volante delle forze dell'ordine.
	\item Camion dei vigili del fuoco = Autopompa.
	\item Ambulanza.
	\item Veicolo su rotaia = Tram.
\end{itemize}
